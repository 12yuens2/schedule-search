\documentclass{article}
\usepackage{color}
\usepackage{tikz}
\usepackage{float}
\usepackage{tabularx}
\usepackage{amsmath}
\usepackage{amssymb}
\usepackage{listings}
\usepackage{enumitem}
\usepackage{syntax}
\usepackage{csquotes}
%\usepackage[backend=er]{biblatex}
%\addbibresource{references.bib}

\usepackage{tikz}
\usetikzlibrary{automata,positioning}

\definecolor{dkgreen}{rgb}{0,0.6,0}
\definecolor{gray}{rgb}{0.5,0.5,0.5}
\definecolor{mauve}{rgb}{0.58,0,0.82}


\lstset{frame=tb,
  numbers=left,
  stepnumber=1,
  language=Java,
  aboveskip=3mm,
  belowskip=3mm,
  showstringspaces=false,
  columns=flexible,
  basicstyle={\small\ttfamily},
  numberstyle=\color{gray},
  keywordstyle=\color{blue},
  commentstyle=\color{dkgreen},
  stringstyle=\color{mauve},
  breaklines=true,
  breakatwhitespace=true,
  tabsize=2,
  moredelim=**[is][\color{red}]{@}{@},
}

\setlength{\grammarindent}{12em}

%\renewcommand{\lstlistingname}{Algorithm}
%\newcommand{\tablerow}[4]{ #1 & #2 & #3 & #4\\}
\newcommand{\n}[0]{\\[\baselineskip]}
%\newcommand{\qa}[2]{\textbf{Q:} #1 \\ \textbf{A:} #2}
%\newcommand{\argument}[4]{\textbf{#1:} #2 \\ \textbf{#3:} #4}

\title{CS4202 Computer Architecture - Process Scheduling for Heterogeneous Systems}
\author{140011146}

\begin{document}

\maketitle

\section{Introduction}
In this practical, a modern scheduling approach for CPUs is looked into, discussing how the approach works and where the approach performs well or poorly. 
\n
Next, a genetic algorithm was implemented to explore the search space of possible schedules with the GEM5 simulator on two benchmarks. The schedules produced by the genetic algorithm are then compared with both the default approach of the simulator and from a completely random schedule to gain an understanding of the scheduling optimisation space.

\section{Task 1}

\section{Task 2}

\subsection{Methodology}
A genetic algorithm was chosen instead of some form of hill-climbing algorithm
- no obvious direction for climbing  higher, unclear how to change bits except going through all, which would be too many combinations
- easy to fall into local minimum 
- eg if only change lower half of schedule, not exploring changing the top half
- randomly restart top half to escape minima doesn't work because affects entire schedule
- ga with crossover reproduction allows exploring different "better" top halves of the schedule
- mutation avoids local minima

- issue with ga is a bit slow to converge and may miss some maxima

- because we don't know how schedule affects time, unsure if crossover/selection/mutation strategies are suitable

Improvements on GA:
- increased population
- crossover only top halves of parents
- increased mutation due to local minima
- increased/decreased tournament selection size
- less or no elites

Increased population due to stagnation at local maxima for more diversity and less likely to choose similar parents in population

Increased mutation also due to local maxima. Perhaps crossover is not effective because if the schedule is good, only some portion of the top half of the binary is used, so low mutation rate where mutation happens in lower half has no effect.

-> Change crossover to must be halfway point or above? -> Prevent low crossover point which has little effect

Elites introduced to keep good schedules, number of elites randomised so not always the same 

Smaller tournament size to allow weak individuals to increase more diversity
Larger tournament size means more likely to have strong individuals for faster convergence

Optimise size of schedule as much as possible to prevent entire lower half being unused


- if schedule is too short, then the simulator will fail and return 0 for that fitness -> this allows removing potential bad schedules

\subsection{Experimentation}


\subsection{Results}

\subsection{Evaluation}



\end{document}



